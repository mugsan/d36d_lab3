

\documentclass[10pt, titlepage, oneside, a4paper]{article}
\usepackage[T1]{fontenc}
\usepackage[utf8]{inputenc}
\usepackage[swedish]{babel}
\usepackage{amssymb, graphicx, fancyhdr}
\usepackage{hyperref}
\addtolength{\textheight}{20mm}
\addtolength{\voffset}{-5mm}
\renewcommand{\sectionmark}[1]{\markleft{#1}}

\newcommand{\Section}[1]{\section{#1}\vspace{-8pt}}
\newcommand{\Subsection}[1]{\vspace{-4pt}\subsection{#1}\vspace{-8pt}}
\newcommand{\Subsubsection}[1]{\vspace{-4pt}\subsubsection{#1}\vspace{-8pt}}
	


\def\typeofdoc{Laborationsrapport}
\def\course{S0014D}
\def\pretitle{Laboration 3}
\def\title{jobQuery - En applikation för den arbetslöse.}
\def\name{Magnus Björk}
\def\username{magbjr-3}
\def\email{\username{}@student.ltu.se}
\def\graders{Patrik Holmlund}
\def\university{Luleå Tekniska Universitet}


\def\fullpath{\raisebox{1pt}{$\scriptstyle \sim$}\username/\path}


\begin{document}
	\begin{titlepage}
		\thispagestyle{empty}
		\begin{large}
			\begin{tabular}{@{}p{\textwidth}@{}}
				\textbf{\university \hfill \today} \\
				\textbf{\typeofdoc} \\
			\end{tabular}
		\end{large}
		\vspace{10mm}
		\begin{center}
			\LARGE{\pretitle} \\
			\huge{\textbf{\course}}\\
			\vspace{10mm}
			\LARGE{\title} \\
			\vspace{15mm}
			\begin{large}
				\begin{tabular}{ll}
					\textbf{Namn} & \name \\
					\textbf{E-mail} & \texttt{\email} \\
				\end{tabular}
			\end{large}
			\vfill
			\large{\textbf{Handledare}}\\
			\mbox{\large{\graders}}
		\end{center}
	\end{titlepage}

	\lfoot{\footnotesize{\name, \email}}
	\rfoot{\footnotesize{\today}}
	\lhead{\sc\footnotesize\title}
	\rhead{\nouppercase{\sc\footnotesize\leftmark}}
	\pagestyle{fancy}
	\renewcommand{\headrulewidth}{0.2pt}
	\renewcommand{\footrulewidth}{0.2pt}

	\pagenumbering{roman}
    \tableofcontents
	
	\newpage

	\pagenumbering{arabic}

	\setlength{\parindent}{0pt}
	\setlength{\parskip}{10pt}

	\section{Introduktion}
		Laborationsuppgiften var att programmera ett nätverksspel. För att kunna utföra denna uppgift var man tvungen att besitta kunskap inom följande:
		
        \begin{itemize}
            \item \textbf{Sockets} \\
            Spelet skulle innehålla kommunikation över nätverket, detta sköts av diverse sockets:
            	\begin{itemize}
            		\item \textbf{DatagramSocket} \\Hanterar trafik av protokollet UDP, kan skicka unicast till en annan DatagramSocket eller multicast genom att skicka till en MulticastSocket. \textit{Önskemål om förflyttning från klient till server.}
            		
            		\item \textbf{MulticastSocket} \\Hanterar inkommande multicast-trafik av protokollet UDP. En MulticastSocket går med i en multicast-grupp för att få ta del av dess trafik. \textit{Uppdateringar av spelarnas position från server till klienter.)}
            		\item \textbf{ServerSocket}
            		\item \textbf{Socket}
            	\end{itemize}
            
            \item \textbf{InetAddress}
            	\begin{itemize}
            		\item \textbf{Ipv4}
            		\item \textbf{Ipv6}
            	\end{itemize}
            	
			\item \textbf{Threads}            
        \end{itemize}
	\section{Metod}
	\section{Resultat}
	\section{Diskussion}
    
    
\end{document}
